\documentclass[a4paper, 12pt]{report}

%\usepackage[latin1]{inputenc} % saisie caractéres accentués clavier
%\usepackage[T1]{fontenc} % police compatible franéais
%\usepackage[francais]{babel} % caractéres spéciaux en franéais

%\usepackage{amsmath}
%\usepackage{amssymb}
%\usepackage{mathrsfs}
%\usepackage{amsfonts}

\usepackage[utf8]{inputenc}
\usepackage{amsmath}
\usepackage{amsfonts}
\usepackage{breqn}
\usepackage[parfill]{parskip}


\usepackage{graphicx}

%\renewcommand{\thepart}{\arabic{part}}
\renewcommand{\thesection}{\arabic{section}}
\renewcommand{\thesection}{\thepart.\arabic{section}}

\begin{document}

\begin{titlepage}

\newcommand{\HRule}{\rule{\linewidth}{0.5mm}} % Defines a new command for the horizontal lines, change thickness here

\center % Center everything on the page

%----------------------------------------------------------------------------------------
%   HEADING SECTIONS
%----------------------------------------------------------------------------------------

\textsc{\LARGE Université catholique de Louvain}\\[0.5cm] % Name of your university/college
\textsc{\Large Ecole de physique}\\[1.5cm] % Major heading such as course name
\textsc{\Large Simulation numérique en physique [\textsc{\normalsize LPHY2371}]}\\[0.5cm]
 % Minor heading such as course title

%----------------------------------------------------------------------------------------
%   TITLE SECTION
%----------------------------------------------------------------------------------------

\HRule \\[0.4cm]
{ \huge \bfseries Equation d'Advection-Diffusion et Prédictibilité}\\[0.4cm] % Title of your document
\HRule \\[1.5cm]

%----------------------------------------------------------------------------------------
%   AUTHOR SECTION
%----------------------------------------------------------------------------------------

\begin{minipage}[t]{0.6\textwidth}
\begin{flushleft} \large
\begin{tabbing}
\emph{Auteurs:}\\
Arnaud \hspace{0.2cm}\= \textsc{Schils}\\ % Your name
Valéry \> \textsc{Materne}
\end{tabbing}
\end{flushleft}
\end{minipage}
~
\begin{minipage}[t]{0.6\textwidth}
\begin{flushright} \large
\emph{Enseignant:} \\
Pr. Michel \textsc{Crucifix} % Supervisor's Name
\end{flushright}
\end{minipage}\\[1.5cm]

% If you don't want a supervisor, uncomment the two lines below and remove the section above
%\Large \emph{Author:}\\
%John \textsc{Smith}\\[3cm] % Your name

%----------------------------------------------------------------------------------------
%   DATE SECTION
%----------------------------------------------------------------------------------------

{\large Décembre 2016}\\[1.5cm] % Date, change the \today to a set date if you want to be precise

%----------------------------------------------------------------------------------------
%   LOGO SECTION
%----------------------------------------------------------------------------------------

\includegraphics[width=4cm]{Logo_UCL_SCIENCES.jpg}\\[1cm] % Include a department/university logo - this will require the graphicx package

%----------------------------------------------------------------------------------------

\vfill % Fill the rest of the page with whitespace

\end{titlepage}

\part{Exercice d'examen 1}

\section*{Introduction}


\section{Fournissez un schéma numérique explicite d'ordre $\mathcal{O}(h+k)$.
Donnez-en le stencil. Nous pouvons introduire les facteurs $\lambda_d = Dk/h^2$
et $\lambda_b = bk$. Quelles importances ces facteurs ont-ils pour la condition
de stabilité?}

\section{Fournissez la solution analytique de l'équation, ainsi que la relation
de dispersion. Discutez brièvement les cas particuliers déjà vus au cours (D=0, b=0, etc.)}

L'équation aux dérivées partielles à résoudre est la suivante:

\begin{equation}
D \frac{\partial^2 u(x,t)}{\partial x^2} - a \frac{\partial u(x,t)}{\partial x} - b u(x,t) = \frac{\partial u(x,t)}{\partial t}
\label{eq:adv_diff}
\end{equation}

où $D, a$, $b$ sont des constantes et $D$ est positive. Dans la suite de ce texte les dépendances en $x$ et $t$
de la fonction $u$ ne seront pas toujours mentionnées explicitement. Les conditions
aux bords suivantes sont imposées:

\begin{equation}
  \left \{
  \begin{aligned}
    u(x,0) = g(x)\\
    u(0,t) = u(l,t) = 0
  \end{aligned}
  \right.
\end{equation}

où $l > 0$. La solution $u$ est donc recherchée dans le domaine:

\begin{equation}
  \left \{
  \begin{aligned}
    t \ge 0\\
    0 < x < l
  \end{aligned}
  \right.
\end{equation}

L'Equation~\ref{eq:adv_diff} peut-être résolue par la méthode de séparation de
variables. La solution $u$ est supposée être de la forme:

\begin{equation}
  u(x,t) = v(x) w(t)
\end{equation}

En injectant cette forme de $u$ dans l'Equation~\ref{eq:adv_diff} on obtient:

\begin{equation}
D w(t) \frac{\partial^2 v(x)}{\partial x^2} - a w(t) \frac{\partial v(x)}{\partial x} - b v(x) w(t) = v(x) \frac{\partial w(t)}{\partial t}
\end{equation}

En divisant cette équation par $v(x) w(t)$ on obtient:

\begin{equation}
  \frac{D}{v} \frac{\partial^2 v}{\partial x^2} - \frac{a}{v} \frac{\partial v}{\partial x} -b = \frac{1}{w} \frac{\partial w}{\partial t} \equiv C_1
\end{equation}

Chaque partie de l'équation est en effet égale à une constante $C_1$ puisque
la partie gauche ne dépend que de $x$ et la partie droite ne dépend que de $t$.
L'équation peut maintenant être résolue en résolvant séparément la partie
qui dépend du temps $t$ et la partie qui dépend de la position $x$. Pour la
partie dépendante du temps on a:

\begin{align}
\frac{1}{w} \frac{\partial w}{\partial t} = C_1\\
\frac{\partial w}{\partial t} = C_1 w\\
w(t) = C_{2} e^{C_1 t}
\end{align}

où $C_{2}$ est une constante. Pour la partie dépendante de la position $x$
on a:

\begin{align}
  \frac{D}{v} \frac{\partial^2v}{\partial x^2} - \frac{a}{v} \frac{\partial v}{\partial x} = C_1 + b\\
  D \frac{d^2v}{d x^2} - a \frac{d v}{d x} - (C_1 + b) v = 0
\end{align}

C'est une équation différentielle linéaire homogène du 2ème ordre. La solution à
son équation dépend alors du polynôme caractéristique de l'équation:

\begin{align}
  D r^2 - a r - (C_1 + b) = 0\\
  \rho = a^2 + 4 D (C_1 + b)
\end{align}

Les deux racines de ce polynôme sont:

\begin{equation}
   r_{1,2} = \frac{a \pm \sqrt{\rho}}{2 D}
 \end{equation}

En fonction du signe de $\rho$ la solution de l'équation peut avoir trois formes.

\underline{Si $\rho > 0$}

\begin{equation}
  v(x) = C_3 e^{r_1 x} + C_4 e^{r_2 x}
\end{equation}

où $C_3$ et $C_4$ sont des constantes. En utilisant la condition au bord
$u(0,t) = 0 = v(0) w(t) \implies v(0) = 0$ on a:

\begin{equation}
  C_4 = -C_3 \implies  v(x) = C_3 (e^{r_1 x} - e^{r_2 x})
\end{equation}

Et en utilisant la condition au bord $u(l,t) = 0 = v(l) w(t) \implies v(l) = 0$ on a:

\begin{equation}
  C_3 (e^{r_1 l} - e^{r_2 l}) = 0 \implies C_{3} = 0 \implies v(x) = 0
\end{equation}

Cette solution n'est donc pas intéressante par rapport à nos conditions aux bords.

\underline{Si $\rho = 0, r_1 = r_2 \equiv r$}

\begin{equation}
  v(x) = (C_3 + C_4 x) e^{r x}
\end{equation}

où $C_3$ et $C_4$ sont des constantes. En utilisant la condition au bord
$u(0,t) = 0 = v(0) w(t) \implies v(0) = 0$ on a:

\begin{equation}
  C_3 = 0 \implies   v(x) = C_4 x e^{r x}
\end{equation}

Et en utilisant la condition au bord $u(l,t) = 0 = v(l) w(t) \implies v(l) = 0$ on a:

\begin{equation}
  C_4 l e^{r l} = 0 \implies C_4 = 0 \implies v(x) = 0
\end{equation}

Cette solution n'est donc pas intéressante par rapport à nos conditions aux bords.

\underline{Si $\rho < 0$}

On définit
\begin{equation} 
r_{1,2} \equiv \alpha \pm i \beta
\end{equation}

avec 

\begin{eqnarray}
  \alpha & = & \frac{a}{2 D}\;,\\
  \beta & = & \frac{\sqrt{-a^{2} - 4 D (C_{1} + b)}}{2 D}\;.\label{beta}
\end{eqnarray}
 
On a alors comme solution pour $v(x)$ :

\begin{equation}
  v(x) = (C_3 \cos(\beta x) + C_4 \sin(\beta x)) e^{\alpha x}\;.
\end{equation}

En utilisant la condition au bord $u(0,t) = 0 = v(0) w(t) \implies v(0) = 0$ on a:

\begin{equation}
  C_3 = 0 \implies v(x) = C_4 \sin(\beta x) e^{\alpha x}
\end{equation}

Et en utilisant la condition au bord $u(l,t) = 0 = v(l) w(t) \implies v(l) = 0$ on a:

\begin{equation}
  \sin(\beta l) = 0 \implies \beta l = m \pi \implies \beta = \frac{m \pi}{l}, m \in \mathbb{N}^{\ast}\;.
\end{equation}

On notera que $\beta > 0$ car $\rho < 0$ et $D > 0$. \\
La solution générale pour la partie de l'équation dépendante de la position est
donc:

\begin{equation}
  v(x) = \sum_{m=1}^{\infty} C_{4_m} \sin \left (\frac{m \pi x}{l} \right ) e^{\alpha x}
\end{equation}

où les $C_{4_m}$ sont des constantes. Utilisons maintenant la condition au bord
$u(x,0) = g(x)$ afin de déterminer les valeurs de ces constantes $C_{4_m}$:

\begin{equation}
  u(x,0) = v(x) w(0) = v(x) C_{2} = g(x)\;.
\end{equation}

On a alors,

\begin{equation}
  g(x) = \sum_{m=1}^{\infty} C_{4_m}C_{2} \sin \left (\frac{m \pi x}{l} \right ) e^{\alpha x}\;.
\end{equation}

En définissant $C_m = C_{4_m}C_{2}$ on obtient:

\begin{equation}
  g(x) = \sum_{m=1}^{\infty} C_m \sin \left (\frac{m \pi x}{l} \right ) e^{\alpha x}
\end{equation}

On voit dès lors que les coefficients $C_m$ sont obtenus
en projetant la fonction $g(x)e^{-\alpha x}$ sur la base des fonctions $\sin \left (\frac{m \pi x}{l} \right )$:

\begin{equation}
  C_m = \frac{2}{l} \int_0^l g(x) \sin \left (\frac{m \pi x}{l} \right ) e^{-\alpha x} dx
\end{equation}

%Puisque le terme de la somme pour $m=0$ est égal à zéro, et que la fonction
%sinus est impaire:

%\begin{align}
%  \tilde{C}_{4_{(-m)}} = \frac{2}{l} \int_0^l g(x) \sin \left (\frac{-m \pi x}{l} \right ) e^{-\alpha x} dx = -\tilde{C}_{4_m}\\
%  \implies \tilde{C}_{4_{(-m)}} \sin \left (\frac{-m \pi x}{l} \right ) e^{\alpha x} = \tilde{C}_{4_m} \sin \left (\frac{m \pi x}{l} \right ) e^{\alpha x}
%\end{align}

%Et donc:

%\begin{equation}
%  g(x) = \sum_{m=1}^{\infty} 2 \tilde{C}_{4_m} \sin \left (\frac{m \pi x}{l} \right ) e^{\alpha x}
%\end{equation}

%En définissant $C^'_{4_m} = 2 \tilde{C}_{4_m}$ on peut écrire:

%\begin{equation}
%  g(x) = \sum_{m=1}^{\infty} C^'_{4_m} \sin \left (\frac{m \pi x}{l} \right ) e^{\alpha x}
%\end{equation}

%avec

%\begin{equation}
%  C^'_{4_m} =  \frac{2}{l} \int_0^l g(x) \sin \left (\frac{m \pi x}{l} \right ) e^{-\alpha x} dx
%\end{equation}

La fonction $u(x,t)$ peut alors s'écrire:

\begin{equation}
u(x,t) = \sum_{m=1}^{\infty} C_m \sin \left (\frac{m \pi x}{l} \right ) e^{\alpha x} e^{C_{1_m} t}
\end{equation}

Nous devons maintenant déterminer l'expression de la constante $C_{1_m}$ qui dépend de $\beta$ et donc de $m$. En partant de l'expression de $\beta$ \eqref{beta}, on obtient :

\begin{equation}
  C_{1_m}= \frac{-a^2 - 4D^{2}\beta^{2}}{4D}-b
\end{equation}
et avec $\beta = \frac{m \pi}{l}$, on a :
\begin{equation}
  C_{1_m}= - \frac{a^2}{4D} - \frac{D m^2 \pi^2}{l^2}-b\;.
\end{equation}

En injectant l'expression de $C_{1_m}$ dans celle de $u$ on a alors :

\begin{equation}
u(x,t) = \sum_{m=1}^{\infty} C_m \sin \left (\frac{m \pi x}{l} \right ) \exp \left ( \frac{a x}{2D} + \left ( - \frac{a^2}{4D} - \frac{D m^2 \pi^2}{l^2}-b \right ) t \right )
\end{equation}

\begin{equation}
  \boxed{u(x,t) = \sum_{m=1}^{\infty} C_m \sin \left (\frac{m \pi x}{l} \right ) \exp \left ( \frac{a}{2D} \left (x - \frac{a}{2} t \right ) - \left (\frac{D m^2 \pi^2}{l^2}+b \right ) t \right )\\
}
\end{equation}

avec

\begin{equation}
  \boxed{C_m = \frac{2}{l} \int_0^l g(x) \sin \left (\frac{m \pi x}{l} \right ) \exp \left ( -\frac{a}{2D} x \right ) dx}\;.
\end{equation}

\section*{Conclusion}

\part{Prédictibilité}


\section*{Introduction}


\section{todo}


\section*{Conclusion}


\end{document}
