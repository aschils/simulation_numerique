\documentclass{report}

\usepackage[utf8]{inputenc} % saisie caractères accentués clavier
\usepackage[T1]{fontenc} % police compatible français
\usepackage{lmodern}
\usepackage[francais]{babel} % caractères spéciaux en français

\usepackage{amsmath}
\usepackage{amssymb}
\usepackage{mathrsfs}
\usepackage{physics}

\usepackage{graphicx}

\renewcommand{\thepart}{\arabic{part}}
\renewcommand{\thesection}{\arabic{section}}
\renewcommand{\thesection}{\thepart.\arabic{section}}


\begin{document}

\begin{titlepage}

\newcommand{\HRule}{\rule{\linewidth}{0.5mm}} % Defines a new command for the horizontal lines, change thickness here

\center % Center everything on the page
 
%----------------------------------------------------------------------------------------
%   HEADING SECTIONS
%----------------------------------------------------------------------------------------

\textsc{\LARGE Université catholique de Louvain}\\[0.5cm] % Name of your university/college
\textsc{\Large Ecole de physique}\\[1.5cm] % Major heading such as course name
\textsc{\Large Simulation numérique en physique [\textsc{\normalsize LPHY2371}]}\\[0.5cm]
 % Minor heading such as course title

%----------------------------------------------------------------------------------------
%   TITLE SECTION
%----------------------------------------------------------------------------------------

\HRule \\[0.4cm]
{ \huge \bfseries Méthodes spectrales}\\[0.4cm] % Title of your document
\HRule \\[1.5cm]
 
%----------------------------------------------------------------------------------------
%   AUTHOR SECTION
%----------------------------------------------------------------------------------------

\begin{minipage}[t]{0.6\textwidth}
\begin{flushleft} \large
\begin{tabbing}
\emph{Auteurs:}\\
Valéry \hspace{0.2cm}\= \textsc{Materne}\\ % Your name
Arnaud \> \textsc{Schils}
\end{tabbing}
\end{flushleft}
\end{minipage}
~
\begin{minipage}[t]{0.6\textwidth}
\begin{flushright} \large
\emph{Enseignant:} \\
Pr. Bernard \textsc{Piraux} % Supervisor's Name
\end{flushright}
\end{minipage}\\[1.5cm]

% If you don't want a supervisor, uncomment the two lines below and remove the section above
%\Large \emph{Author:}\\
%John \textsc{Smith}\\[3cm] % Your name

%----------------------------------------------------------------------------------------
%   DATE SECTION
%----------------------------------------------------------------------------------------

{\large Décembre 2016}\\[1.5cm] % Date, change the \today to a set date if you want to be precise

%----------------------------------------------------------------------------------------
%   LOGO SECTION
%----------------------------------------------------------------------------------------

\includegraphics[width=4cm]{Logo_UCL_SCIENCES.jpg}\\[1cm] % Include a department/university logo - this will require the graphicx package
 
%----------------------------------------------------------------------------------------

\vfill % Fill the rest of the page with whitespace

\end{titlepage}


\part{Exercice d'introduction : 3 méthodes spectrales}

\section{Introduction}
Nous considérons l'équation différentielle suivante :

\begin{equation}\label{eq:main}
u_{xx} + u_x - 2u + 2= 0
\end{equation}

sur le domaine $-1\leq x \leq 1$ et avec les condition aux frontières $u(-1)=u(1)=0$.

Nous souhaitons approximer la solution analytique exacte de cette équation différentielle :

\begin{equation}
u(x) = 1- \frac{\sinh(2)e^{x}+\sinh(1)e^{-2x}}{\sinh(3)}
\end{equation}

par un développement tronqué de polynôme de Tchebychev :

\begin{equation}\label{eq:app}
v(x) = \sum_{k=0}^N a_k T_k(x) \;.
\end{equation}

\section{propriété des polynômes de Tchebychev}

Le polynôme de Tchebychev de degré $n$, $T_n(x)$, est défini par
\begin{eqnarray}
T_n(x) &=& \cos (n\arccos(x))\;,\\
T_n(\pm1) &=& (\pm1)^n\;.
\end{eqnarray}

Relation d'orthogonalité :

\begin{equation}
\int_{-1}^{1} T_n T_m (1-x^2)^{-1/2} dx = \frac{\pi}{2} c_n \delta_{nm}
\end{equation}
avec $c_0=2$ et $c_n = 1$ pour $n>0$.

Relations de récurrence :

\begin{eqnarray}
T_{n+1}(x) & = & 2xT_n(x) - T_{n-1}(x) \ \ \ \ \ , n \geq 1\;,\\
2T_n (x) & = & \frac{T_{n+1}'(x)}{n+1} - \frac{T_{n-1}'(x)}{n-1} \ \ \ \ , n \geq 2\;.\label{eq:rec}
\end{eqnarray}

\section{Calcul des coefficients du développement du résidu}

Nous injectons la série tronquée \eqref{eq:app} dans l'équation différentielle de départ \eqref{eq:main} et nous définissons le résultat comme le résidu. Nous le redéfinissons selon une nouvelle série tronquée de polynôme de Tchebychev:

\begin{eqnarray}
R(x) & = & v_{xx} + v_x - 2v + 2 \\
& = & \sum_{k=0}^N A_{k} T_{k}(x)\;. \label{eq:reste}
\end{eqnarray}

Nous calculons ensuite ces nouveaux coefficients $A_{k}$ en fonction des $a_{k}$. Pour ce faire nous commençons par exprimer les dérivées de $v$ en fonction des polynômes de Tchebychev.

On recherche les formes suivantes : 

\begin{eqnarray}
v_{x}(x) = \sum_{k=0}^N b_k T_k(x) \label{eq:vxb}\\
v_{xx}(x) = \sum_{k=0}^N c_k T_k(x) \\
\end{eqnarray}

Pour ce faire, on part de l'équation \eqref{eq:app}, on a :

\begin{equation}
v_{x}(x) = \sum_{k=0}^N a_k T_{k}'(x)\;.\label{eq:vxa}
\end{equation}

On doit trouver l'expression des $T_{k}'(x)$ en fonction des $T_{k}(x)$ pour $k>0$. Pour ce faire, on utilise la relation de récurrence \eqref{eq:rec}. Si on pose $k=n+1$, on obtient :

\begin{equation}
T_{k}'(x)  = k\left(2T_{k-1}(x)+\frac{T_{k-2}'(x)}{k-2}\right)\;.
\end{equation}

En la réinjectant dans son terme de droite, par récurrence, on obtient deux cas :

\underline{k impair}
\begin{equation}
T_{k}'(x)  = 2k\left(T_{k-1}(x)+T_{k-3}(x)+\cdots+T_{4}(x)+T_{2}(x)+T_{0}(x)/2\right)\;,\label{eq:kimpair}
\end{equation}

\underline{k pair} 

\begin{equation}
T_{k}'(x)  = 2k\left(T_{k-1}(x)+T_{k-3}(x)+\cdots+T_{5}(x)+T_{3}(x)+T_{1}(x)\right)\;.\label{eq:kpair}
\end{equation}

On a donc

\begin{equation}
\begin{pmatrix}
 b_{0}\\ 
 b_{1}\\ 
 \vdots\\ 
 b_{N-1}\\ 
 b_{N}
\end{pmatrix} 
= D \cdot \begin{pmatrix}
 a_0\\ 
 a_1\\ 
 \vdots\\ 
 a_{N-1}\\ 
 a_{N}
\end{pmatrix}
\end{equation}

où $D$ est la matrice dérivée.

On va maintenant définir cette matrice. En égalant les équations \eqref{eq:vxb} et \eqref{eq:vxa} et en remplaçant les expressions de $T_{k}'(x)$ par \eqref{eq:kimpair} et \eqref{eq:kpair}, on obtient les valeurs de $b_{k}$ suivantes :

\underline{k impair} 

\begin{equation}
b_{k} = 2\sum_{n=k-1}^{\frac{N-1}{2}}(2n)a_{2n} \;, \\\\\ si \ N \ est \ impair
\end{equation}

\begin{equation}
b_{k} = 2\sum_{n=k-1}^{\frac{N}{2}-1}(2n) a_{2n} \;, \\\\\ si \ N \ est \ pair
\end{equation}


\underline{k pair} 

 $k\geq2$
\begin{equation}\label{eq:bpairNimpair}
b_{k} = 2\sum_{n=k/2}^{\frac{N-1}{2}}(2n+1)a_{2n+1} \;, \\\\\ si \ N \ est \ impair
\end{equation}

\begin{equation}\label{eq:bpairNpair}
b_{k} = 2\sum_{n=k/2}^{\frac{N}{2}-1}(2n+1)a_{2n+1} \;, \\\\\ si \ N \ est \ pair
\end{equation}

Pour le cas $k=0$, $b_{0}$ est égal à la moitié de \eqref{eq:bpairNimpair} ou \eqref{eq:bpairNpair} selon la parité de $N$.

Il est maintenant possible de définir les éléments de matrice $D$ :

\begin{equation}
d_{ij} = 2j\alpha_{i}\beta_{ij}\gamma_{ij} \;, 
\end{equation}

avec

\begin{align}
\alpha_{i} &= 
  \begin{cases}
    \frac{1}{2} & \text{si $i =0$} \\
1 & \text{sinon}
  \end{cases}\;,\\
\beta_{ij} &=
  \begin{cases}
    1 & \text{si i et j sont de parité opposée}\\
0 & \text{sinon}
  \end{cases}\;,\\
 \gamma_{ij} &=
  \begin{cases}
    1 & \text{si i < j}\\
   0 & \text{sinon}
  \end{cases}\;.
\end{align}


On a implémenté cette matrice dans Matlab, valide pour tout $N$, voici le cas $N=4$ :

\begin{equation}
\begin{pmatrix}
0 & 1 & 0 & 3 & 0\\ 
0 & 0 & 8 & 0 & 8\\ 
0 & 0 & 0 & 6 & 0\\ 
0 & 0 & 0 & 0 & 8\\ 
0 & 0 & 0 & 0 & 0
\end{pmatrix} \;.
\end{equation}

\section{Méthode Tau}

\section{Méthode Galerkin}

\section{Méthode Collocation}

\section{Méthode aux différences finies}

\section{Conclusion}

\section{vais dormir}

 \part{Exercice sur l'équation de Schrödinger}
\setcounter{section}{0}
 
\section{Introduction}


\section{section 1}


\section*{Conclusion}


\end{document}
